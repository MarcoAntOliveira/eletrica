\section{Analise Númerica}
\label{sec:analise_numerica}
A análise será conduzida sobre um circuito LC alimentado por uma fonte de tensão alternada, composto pelos elementos descritos na Tabela  Cada componente é representado por seu símbolo convencional, acompanhado da equação que rege seu comportamento dinâmico e sua respectiva unidade de medida.

\begin{table}[H]
\centering
\begin{tabular}{@{}ll@{}}
\toprule
\textbf{Parâmetro} & \textbf{Valor} \\
\midrule
Tensão RMS da fonte $V_{s,\text{RMS}}$ & \SI{224}{\volt} \\
Impedância da linha $Z_{\text{lin}}$ & $0.24 + 0.029i\ \Omega$ \\
Potência ativa da carga 1 $P_1$ & \SI{1140}{\watt} \\
Potência aparente da carga 1 $|S_1|$ & \SI{1580}{\voltampere} \\
Potência reativa da carga 2 $Q_2$ & \SI{1580}{\var} \\
Fator de potência da carga 2 $fp_2$ (atraso) & 0{,}76 \\
Potência ativa da carga 3 $P_3$ & \SI{1430}{\watt} \\
Potência reativa da carga 3 $Q_3$ & \SI{1405}{\var} \\
\bottomrule
\end{tabular}
\caption{Parâmetros das cargas e da fonte}

\begin{center}
\begin{circuitikz}[american]
    % Fonte de tensão senoidal
    \draw (0,0)
        to[sV, l=$V_s$] (0,4)
        to[R, l=$Z_{\text{linha}}$] (2,4)
        -- (4,4);

    % Dividindo para as três cargas
    \draw (4,4)
        to[short] (4,3.5);
    \draw (4,3.5)
        to[R, l=$Z_1$] (4,2)
        -- (0,2)
        -- (0,0);

    \draw (4,3.5)
        to[R, l=$Z_2$] (5.5,2)
        -- (5.5,0)
        -- (0,0);

    \draw (4,3.5)
        to[R, l=$Z_3$] (7,2)
        -- (7,0)
        -- (0,0);
\end{circuitikz}


A modelagem esquemática do circuito é apresentada a seguir:


A impedância do indutor é dada por:
\[
Z_L = j\omega L
\]
e a impedância do capacitor é:
\[
Z_C = \frac{1}{j\omega C}
\]

Aplicando o divisor de tensão, a tensão sobre o capacitor \( V_C(t) \) é dada por:
\[
V_C(t) = V_{AC}(t) \cdot \frac{Z_C}{Z_L + Z_C}
\]

Substituindo as impedâncias, temos:
\[
V_C(t) = V_0 \sin(\omega t) \cdot \frac{\frac{1}{j\omega C}}{j\omega L + \frac{1}{j\omega C}}
\]

Multiplicando numerador e denominador por \( j\omega C \), obtemos:
\[
V_C(t) = V_0 \sin(\omega t) \cdot \frac{1}{1 - \omega^2 LC}
\]
\end{center}

\subsection*{Observações}
\begin{itemize}
    \item Quando \( \omega^2 LC = 1 \), o circuito entra em \textbf{ressonância} e o denominador se anula, indicando uma resposta teórica infinita (na prática, limitada por resistências e perdas).
    \item Para \( \omega < \frac{1}{\sqrt{LC}} \), o capacitor tende a dominar a queda de tensão.
    \item Para \( \omega > \frac{1}{\sqrt{LC}} \), o indutor domina a queda de tensão.
\end{itemize}
\newpage
\section{Especificações da Simulação no LTSpice}
\label{sec:especificacoes}

\subsection*{Objetivo}

Criar um circuito divisor de tensão utilizando impedâncias $Z_1$ e $Z_2$ de forma que a tensão de saída seja, em módulo, maior que a tensão de entrada na frequência de ressonância. O circuito deve ser implementado no LTSpice e simulado de acordo com as práticas realizadas em laboratório.

\subsection*{Simulação em Análise AC}

\begin{itemize}
    \item A análise AC deve conter uma varredura com \textbf{100 pontos por década} (``dec 100'').
    \item A frequência mínima deve ser \textbf{uma década abaixo} da frequência de ressonância $f_0$.
    \item A frequência máxima deve ser \textbf{uma década acima} da frequência de ressonância $f_0$.
    \item Exemplo de sintaxe para $f_0 = 15\,\text{kHz}$:
    \begin{center}
        \verb|.ac dec 100 1500 150000|
    \end{center}
    \item O gráfico da análise AC deve apresentar o módulo da função de transferência $|V_{\text{out}}/V_{\text{in}}|$ em função da frequência.
\end{itemize}

\subsection*{Simulação Transitória (Transient)}

\begin{itemize}
    \item A simulação transitória deve cobrir \textbf{100 ciclos} da frequência de ressonância.
    \item Para $f_0 = 15\,\text{kHz}$, o período é $T = \frac{1}{f_0} \approx 66.6\,\mu s$.
    \item Portanto, o tempo total de simulação deve ser:
    \[
        t_{\text{stop}} = 100 \cdot T = 100 \cdot 66.6\,\mu s = 6.66\,\text{ms}
    \]
    \item A visualização no gráfico deve focar nas \textbf{duas últimas oscilações}. Para isso, configure:
    \begin{itemize}
        \item \textbf{Stop time:} \verb|6.66m|
        \item \textbf{Start saving data:} \verb|6.5m| (ou valor um pouco antes de 6.66 ms)
        \item \textbf{Maximum Timestep:} \verb|66u| (opcional, ajuda na precisão)
    \end{itemize}
    \item Comando final:
    \begin{center}
        \verb|.tran 0 6.66m 6.5m 66u|
    \end{center}
\end{itemize}

\subsection*{Considerações de Modelagem}

\begin{itemize}
    \item O gerador de sinais deve ser modelado com sua \textbf{impedância interna} (por exemplo, uma resistência em série).
    \item Os componentes passivos (indutor e capacitor) devem considerar apenas os \textbf{efeitos parasitas discutidos em laboratório}.
    \item Efeitos como resistência série do indutor ($R_s$) ou ESR do capacitor podem ser incluídos se indicado.
    \item A montagem do esquemático deve seguir o \textbf{modelo visto nas práticas} de laboratório.
\end{itemize}


\section{Resultados}
 \textbf{Equação de ressonância angular:}

   $$
   \omega^2 LC = 1 \Rightarrow \omega = \frac{1}{\sqrt{LC}}
   $$

 \textbf{Condição para a frequência angular $\omega$:}

   $$
   \omega > \frac{1}{\sqrt{LC}}
   $$

   Isso significa que, para o sistema estar **acima da frequência de ressonância**, a frequência angular deve ser maior que o valor de ressonância.

\textbf{Comparação de frequências:}

   $$
   9{,}4248 \times 10^4 < 10 \times 10^5
   $$

   Isso mostra que a frequência de operação (aparentemente próxima de 94,248 kHz) está **abaixo** de uma frequência de referência (possivelmente 1 MHz), então ainda está **abaixo da faixa ressonante**.

Valor encontrado para $\omega^2 LC$:

   $$
   \omega^2 LC = 2{,}25 \times 10^{-2}
   $$

   Isso indica que a condição de ressonância ainda **não foi atingida**, pois o valor ideal seria $\omega^2 LC = 1$. O sistema está operando **abaixo da ressonância**.

---

\subsection*{Conclusões}
\begin{itemize}
    \item  A frequência angular $\omega$ calculada ainda **não atinge a condição de ressonância**, pois $\omega^2 LC < 1$.
    \item  Para que o circuito opere **em ressonância**, é necessário aumentar a frequência ou ajustar os valores de **L** e **C** para que:

           $$
           \omega = \frac{1}{\sqrt{LC}}
           $$
    \item    Como $\omega^2 LC = 2{,}25 \times 10^{-2}$, a frequência está bem **abaixo da ressonância** — o que significa que o circuito pode estar se comportando como **um filtro passa-baixas**, dependendo da configuração do divisor.
\end{itemize}
\newpage

\begin{figure}[h!]
    \centering
    \includegraphics[width=1.0\linewidth]{imagem_simulacao2.png}
    \caption{simulação LTSpice}
    \label{fig:enter-label}
\end{figure}


Após os testes e simulações realizadas, foi possível observar que o circuito LC inicialmente projetado com $L = 10\ \text{mH}$ e $C = 2{,}5\ \text{nF}$ não operava em ressonância na frequência de excitação escolhida de $15\ \text{kHz}$. A frequência de ressonância teórica nesse caso estava em torno de $31{,}9\ \text{kHz}$, o que gerava distorções na forma de onda de saída.

Para corrigir esse desvio, ajustou-se o valor da capacitância para $10\ \text{nF}$, mantendo a indutância em $10\ \text{mH}$. Essa alteração trouxe a frequência de ressonância do circuito para um valor bem próximo de $15\ \text{kHz}$, resultando em uma resposta senoidal amplificada na saída (Vout), característica esperada em um circuito ressonante série.

Na simulação, observou-se que a tensão de saída atinge picos significativamente maiores que a tensão de entrada, demonstrando o fenômeno de ressonância e a capacidade do circuito de amplificar sinais na frequência natural. A forma de onda de saída tornou-se mais limpa e simétrica, validando o dimensionamento correto dos componentes para operação em 15 kHz.

Portanto, conclui-se que a correta escolha dos parâmetros L e C é essencial para garantir o funcionamento ressonante do circuito, principalmente quando a frequência de excitação é fixa.


\begin{figure}[h!]
    \centering
    \includegraphics[width=1.0\linewidth]{imagem_simulação4.png}
    \caption{Nova simulação}
    \label{fig:enter-label}
\end{figure}
